\chapter{Radio Driver for Zolertia with the RN2483 module\label{section:radio}}

% Most of the LoRaWAN features were not implemented or tested since this was out of the scope of this project.

The \emph{Zolertia RE-Mote REV-B} platform does not support \emph{LoRa} out of
the box. This chapter covers the implementation of a radio driver on
\emph{Contiki-NG} for the \emph{RN2483}, a LoRa module from Microchip,
communicating with the \emph{RE-Mote} platform.

This chapter will also give the reader more context about the inner working of
the RN2483 module.
It will help understand the inherent issues with the module
described in chapter~\ref{section:tsch}.

\section{Preliminary work}

Last year \emph{Roald Van Glabbeek}, in the course of his master thesis~\cite{8847137}, 
aiming at doing further research on \emph{energy efficient LoRa multihop networks}, 
started working on the \emph{RN2483} module.
For this purpose he made \emph{RN2483 Shields} adapted to the pin
configuration of the \emph{Zolertia RE-Mote REV-B}. He also started a radio driver
implementation for Contiki-OS\@.

In the previous attempt to adapt \emph{TSCH} for \emph{LoRa}
in~\cite{njomgang_2018}, \emph{Serge Sadja Njomgang} did his experimentation with 
the \emph{LoRaMote} demo platform from \emph{Semtech} using the \emph{SX1272} LoRa 
radio module.
As analyzed in~\cite{8847137}, the Semtech LoRaMotes turned out to be an issue for
further research because they were no longer produced, 
new versions were expensive and memory was a bottleneck.
That is why the \emph{Zolertia RE-Mote} was chosen instead. 
The platform is already available and well maintained in \emph{contiki-ng}
(making the transition faster and the code less error-prone) and the \emph{ETRO
Lab} is already utilizing the platform.

The first part of my work consisted of implementing a reliable
and fully featured radio driver for the \emph{RN2483} module working in the last
version of \emph{contiki-ng}.

\section{RN2483 Module Structure}

Developed by \emph{Microchip} the \emph{RN2483} is a LoRa module operating in
the 433 MHz and 868 MHz Frequency Bands~\cite{microchip:rn2483}. 
The module is specifically designed to work with LoRaWAN compatible networks, 
by including a set of commands designed for seamless integration with the
\emph{LoRaWAN Protocol Stack}

Each communication to and from the module is done via \emph{UART} ASCII command,
making it easy for a human to interact with the module by handwriting commands
on a terminal and reading the response back in a readable format
(see~\ref{fig:pcconn}).
% The module is made for ease of use over performance and power consumption.

\begin{figure}[H] % TODO More info on axis
\centering
\begin{tikzpicture}[auto, thick]
  \node(pc) [server, label=below:{PC}] {};
  \node(ftdi) [draw, rectangle, right=of pc, right=4cm] {USB to UART Adapter};
  \node(module) [draw, rectangle, minimum size=6mm, right=of ftdi, right=2.5cm] (module) {RN2483 Module};

  \path (pc) edge[<->] node[]{USB Connection} (ftdi) ;
  \path (ftdi) edge[->,bend left] node[text width=1cm, align=center]{TX} (module);
  \path (module) edge[->,bend left] node[text width=1cm, align=center]{RX} (ftdi);
\end{tikzpicture}
\caption{Simple communication with the module scenario}
\label{fig:pcconn}
\end{figure}


The module's interface includes three types of commands that enable access to
different functions~\cite{microchip:reference}.

\begin{itemize}
  \item \emph{radio} for the low-level radio commands to access the transceiver
    and PHY settings directly without the LoRaWAN interface overhead.
  \item \emph{mac} to access the LoRaWAN protocol stack configurations and
    commands. This command set will be used less because our implementation
    requires low-level transceiver access.
  \item \emph{sys} for the module-specific configurations such as the module
    GPIOs state, \emph{sleep}, EEPROM memory access, \ldots
\end{itemize}

\section{Testing Setup}

My testing setup was the same as~\cite{8847137}. I used the Zolertia RE-Mote
Rev-B platform, using the CC2538 microcontroller, connected to the
RN2483 breakout board, as schematized in Fig~\ref{fig:schemaconn}. 

The RE-Mote platform has two UART peripheral, the UART1 peripheral is used
to communicate with the module because the platform debugger uses UART0.

The \lstinline{PD0} GPIO is wired to the reset pin to allow
software reset of the module as well as a push-button in case of hard reset.

\begin{figure}[H]
  \centering
  \includegraphics[scale=0.70]{thesis.tex/chapters/driver/fig/conn_diag.pdf}
  \caption{Hardware connection\label{fig:schemaconn}}
\end{figure}

\section{Implementation}

All the files for the driver implementation are available in
\lstinline{/arch/dev/rn2483} and the location of the examples are in
\lstinline{/examples/platform-specific/rn2483}.

\subsection{Structure}

My implementation structure took inspiration from the work 
in~\cite{8847137} and the \emph{RIOT-OS}
project\footnote{\url{https://github.com/RIOT-OS/RIOT}}. 

Following is a listing of the different files I created and their
usage. % \improvement{I don't like this phrase}

\begin{description}
  \item[lora] LoRa radio specific functions declaration, and configuration. 
    This includes functions used to calculate the time-on-air and
    LoRa PHY configuration structure and parameters.
  \item[rn2483-include] contains the declaration both needed by
    \lstinline{rn2483-uart} and \lstinline{rn2483} to avoid cycling inclusion
    that mess with the compiler.
  \item[rn2483-uart] Lower level implementation of the function to directly
    communicate with the module via UART\@.
    It contains hardware-specific implementation for the UART\@.
  \item[rn2483-api] Complete coverage of the RN2483 commands. Using functions
    from this file allows the user of the driver to not write
    each command by hand and use standard structure instead.
  \item[rn2483] Implementation of the Contiki radio device driver. 
    It contains the standardized way of Contiki to use and configure radio
    devices.
    All the radio drivers in Contiki share this structure.
\end{description}

\subsection{Synchronous communication with the module}

The first design challenge is the
synchronous UART message mechanism implementation.
UART communication needs synchronous messaging for every
transmission, configuration, and reception.
Acknowledgements from commands avoid 
clashes when trying to use the module at the same time.
The previous implementation in~\cite{8847137} used fixed time delays to 
wait for the command to full execution on the module. This method 
quickly showed its limitation, each command has its delay, and
fixed delays slow the global execution.

% TODO Maybe sequence diagram with a clash

The UART driver for the CC2538 in Contiki does not have synchronous 
blocking read function out of the box, it assumes the programmer will compute the
responses from UART in processes waiting for an event instead.
Reception works by executing an \emph{input handler} at each interruption
triggered from UART reception.
Contiki set the default input handler with the function
\lstinline{serial_line_input_byte},
specifically designed for serial communication with terminals. 
The function \lstinline{serial_line_input_byte} pass each character to standalone
process (\lstinline{serial_line_process}) and broadcast an event on command
full reception. 
I implemented a custom handler directly treating each message in the
interrupt instead of using processes.

Working with processes and events is impossible in our use case because Contiki 
does not have any mechanism that allows programmers to wait for an event inside a 
function and only allows it inside processes.
Doing everything inside a process is not an option for this project.
The custom handler is designed as a state machine (Fig~\ref{fig:cmdstate}) and 
works with its reception buffer. It takes advantage of the limited response 
format from the RN2483 module to sort each new communication.

\begin{itemize}
  \item Asynchronous LoRa radio messages start with \lstinline{radio rx  }.
  \item \lstinline{radio_tx_ok} indicates the end of the radio transmission. 
  \item \lstinline{ok} indicates the acknowledgement of a command.
\end{itemize}

% Talk about the state machine

I implemented helper functions, to ease the command transmission and
acknowledgement between the RE-Mote and the RN2483.

\begin{description}
  \item[rn2483\_receive\_synch] busy-wait on the command states
    (Fig~\ref{fig:cmdstate}) until reaching the \emph{received} state 
    or the timeout duration has elapsed.
  \item[rn2483\_send\_cmd] a \emph{printf} style command, automatically 
    waiting for the acknowledgement (with the possibility to add custom timeouts).
  \item[rn2483\_raw\_cmd] sending the command from a buffer in argument
    without formatting or waiting for the acknowledgement.
\end{description}

\begin{figure}[H]
\centering
  \begin{tikzpicture}[->,>=stealth',shorten >=1pt,auto,node distance=3.5cm]
  \tikzstyle{every state}=[thick,draw=gray!50,fill=gray!20,draw=none,text=black]

  \node[initial,state] (A)                    {Transmission};
  \node[state]         (B) [above right of=A] {Reception};
  \node[state]         (C) [below of=B]       {Timeout};
  \node[state]         (D) [right of=B]       {Received};

  \path (A) edge [bend right]    node {     } (B)
            edge [bend left]     node {     } (C)
        (B) edge [bend left]     node {     } (C)
            edge                 node {     } (D)
        (C) edge [bend left]     node {     } (A)
        (D) edge [bend right=85] node {     } (A);
\end{tikzpicture}
\caption{The command states\label{fig:cmdstate}}
\end{figure}


\subsection{Transmission}

The radio transmission command has a different pattern from the other commands 
of the module. 
After the acknowledgement of the radio transmission command (\lstinline{radio tx ...}),
the module transmits a second message when the radio communication is done.

\begin{figure}[H]
\centering

\begin{sequencediagram}

\newthread{A}{RE-Mote}{}
\newinst[1]{B}{RN2483}{}
\newinst[3]{C}{Radio Band}{}
\begin{call}{A}{radio tx~\ldots}{B}{radio\_tx\_ok}
  \messdash{B}{ok}{A}
  
  \begin{sdblock}{Radio Transmission}{Transmission time calculated
    in~\ref{eq:tpacket}}
    \begin{call}[4]{B}{}{C}{}
    \end{call}
  \end{sdblock}
\end{call}

\end{sequencediagram}

  \caption{\lstinline{radio tx} command sequence diagram\label{fig:txsequence}}

\end{figure}



Microchip chose to represent the payload as string formatted hex number to
facilitate the manual writing of commands, making the message payload twice as
long as its original content. 
As all the content we will handle is represented as an \emph{8-bit} number array, 
I took the function to convert array to string representation and vice versa
from the RIOT-OS project.  % TODO REF

\subsection{Asynchronous Reception}

The last component of a complete driver implementation is to receive radio
messages.
Reception follows a different scheme from the other commands. 
New message reception can happen at any time, asynchronously, from the moment
we run the command \lstinline{radio rx 0}.
I implemented a one message inbox to tackle this issue (see Fig~\ref{fig:rxstate}).

Every message coming from the module is analyzed.
Radio communication, starting with \lstinline{radio rx }, set a flag to indicate the 
presence of a new message. 
This flag remains until the message is read with the driver.
Messages are received in a string formatted hex number, that we must convert it to an
\emph{8-bit integer array}. 
% TODO Talk about packet buf ?
% We don't store the final result in the same buffer as the one used for
% reception as reading the result 
% This array is stored in a different place than the received message as command
% may be executed between the moment we receive the message and the moment we
% read it.
Also, a timestamp is saved at the reception to keep track of when the
message was received.

\begin{figure}[H]
\centering
  \begin{tikzpicture}[->,>=stealth',shorten >=1pt,auto,node distance=3.5cm]
  \tikzstyle{every state}=[thick,draw=gray!50,fill=gray!20,draw=none,text=black]

  \node[initial,state] (A)              {Empty};
  \node[state]         (B) [right of=A] {Received};

    \path (A) edge [bend right] node[below] {Reception} (B)
          (B) edge [bend right] node[above] {Read} (A);
\end{tikzpicture}
\caption{The RX states that implement a one message inbox\label{fig:rxstate}}
\end{figure}


\subsection{Contiki Radio Driver}

The Contiki WIKI did not provide any information about writing our radio driver
or porting a new radio platform for Contiki.
I based my implementation on the other radio driver implementation already
available on Contiki to understand the driver structure.

To create a Contiki radio driver we have to implement a \lstinline{radio_driver}
structure that must implement the following functions.

\begin{description}
  \item[init] is the first function of the driver executed at startup.
    The configuration of the RN2483 into point-to-point mode takes place here.
  \item[prepare] is the function called before data transmission.
    We can not transmit radio messages while listening for incoming messages, the
    first step of preparation is to exit the radio reception mode with the
    command \lstinline{radio rxstop}.
    The function takes care to convert the data in the command ready to be sent.
  \item[transmit] sends the already formatted command in the prepare function and
    waits for the \lstinline{radio_tx_ok} acknowledgement.
  \item[send] is the combination of \lstinline{prepare} and
    \lstinline{transmit}.
  \item[read], on radio message reception, the function copies the 
    message to the buffer passed in the argument.
  \item[channel\_clear]
  \item[receiving\_packet] should theoretically check if the radio module is
    currently getting a new message from a transmission. This is a function we
    can expect from some platforms with an integrated radio module but here the
    simplicity of the RN2483 shows its limitations and I found no way for the module
    to acknowledge when it receiving a new message.
  \item[pending\_packet] checks the message inbox to see if we received a new
    message.
  \item[on] wake-up the module and starts listening to radio messages.
  \item[off] stop listening to radio messages and put the module to sleep.
  \item[get\_value, set\_value, get\_object, set\_object] are all the functions used
    to configure the radio driver for a different network layer. 
    The parameters we need to implement will be discussed in Chapter~\ref{section:tsch}.
\end{description}

% TODO talk about Poll mode

\subsection{Device Configuration}

The module can be configured through the following structure.
It allows programmers to access LoRa different parameters from the code without
having to interact with the module.

\begin{lstlisting}[language=C]
typedef struct {
  radio_modulating_mode mod;
  radio_channel chan;
  uint32_t freq;
  radio_sf sf;
  radio_bw bw;
  radio_cr cr;
  uint32_t prlen;
  radio_pwr pwr;
  bool crc;
  bool iqi;
  bool explicit_header;
  uint32_t wdt;
} lora_radio_t;
\end{lstlisting}

This structure has a default definition but can be redefined from anywhere 
by redefining \lstinline{RN2483_DEV_CONF}.
The currently used parameters are accessed with \lstinline{RN2483_DEV}.

We will see in Chapter~\ref{section:tsch} how those variables are used for
calculations.

\section{Validation test}

To verify the implementation of the radio driver two examples are available.

\begin{itemize}
  \item The \emph{RN2483 shell} available in
    \lstinline{/example/platform-specific/lora/rn2483} is a tool to interact
    directly with the module.
  \item The \emph{ping-pong} example based on the nullnet example already
    available in \emph{contiki-ng} located in 
    \lstinline{/example/platform-specific/lora/nullnet} demonstrate the driver
    is correctly working with the upper layers of the network stack.
\end{itemize}

% For each Contiki project, \emph{Contiki's Network Protocol stack}, also known
% as \lstinline{NETSTACK}, defines the different OSI layers implementations we 
% want to use in our project (Fig~\ref{fig:netstack}).
Each project is bundled with a \lstinline{project-conf.h}
configuration file in which we can define the parameters of our project. 
To set my LoRa driver as the physical layer implementation we must add the
following line to change the NETSTACK configuration.

\begin{lstlisting}
#define NETSTACK_CONF_RADIO rn2483_radio_driver
\end{lstlisting}

\subsection{Shell}

This example extends the existing shell available in Contiki by writing a set
of commands to test and interact with the RN2483 via UART\@.

Here is the list of the added commands to the Contiki shell.

\begin{lstlisting}[language=none]
'> sys': Send a 'system' command to the RN2483 Module.
'> mac': Send a 'mac' command to the RN2483 Module.
'> radio': Send a 'radio' command to the RN2483 Module.
'> tx': Transmit a radio message with the RN2483.
'> sleep': Put RN2483 into sleep mode.
'> wakeup': Force wakeup of RN2483.
'> rn2483_reset': Hardware Reset of the RN2483 Module.
'> help': Shows this help.
\end{lstlisting}

This was used during the development of the driver as a way to test parameters
as well as the synchronous communication implementation.

\subsection{Ping-Pong\label{section:pingpong}}

I based this example on the \emph{nullnet-broadcast} example to test if two
nodes could communicate using the LoRa driver I implemented when controlled by
the \emph{CSMA} MAC layer instead of directly using the driver API\@.

The example is using the NETSTACK in Fig~\ref{fig:pingpongstack}.

\begin{figure}[H]
\centering
  \begin{tikzpicture}[->,>=stealth',shorten >=1pt,auto,node distance=1.6cm]

  \tikzstyle{comment}=[
    right=2pt,
    font=\small,
    fill=white,
    text=black,
    draw=black,
  ]

  \tikzstyle{every state}=[rectangle,thick,
    draw=black,fill=gray!20,text=black,
    minimum width= 6cm,
    minimum height= 1.20cm
  ]

  \tikzstyle{smallstate}=[rectangle,thick,
    draw=black,fill=gray!10,text=black,
    minimum width= 4cm,
    minimum height= 1.20cm
  ]

  \node[smallstate]         (A)                    {Null Net};
  \node[smallstate]         (B) [below of=A]       {Null Routing};
\begin{scope}[on background layer]
  \node[state, fit=(A)(B)] (AB)                 {};
\end{scope}
  \node[state,below=1cm]         (C) [below of=AB]       {CSMA};
  \node[state]         (D) [below of=C]       {LoRa};

  \node[comment]       at (AB.north west) {Network Layer};
  \node[comment]       at (C.north west) {MAC Layer};
  \node[comment]       at (D.north west) {Physical Layer};
  % \path (A) edge [bend right]    node {     } (B)
  %           edge [bend left]     node {     } (C)
  %       (B) edge [bend left]     node {     } (C)
  %           edge                 node {     } (D)
  %       (C) edge [bend left]     node {     } (A)
  %       (D) edge [bend right=85] node {     } (A);
\end{tikzpicture}
\caption{The network stack used for the ping-pong example\label{fig:pingpongstack}}
\end{figure}


The example is a simple two-way communication between two RE-Motes. 
The first one is sending a request with a number as
payload and wait for the acknowledgement from the second motes sending back
the same payload. 
No other request is sent until the acknowledgement is received.
This is schematized in Fig~\ref{fig:pingpongsequence}.

The two different implementations to upload on two nodes are available in the
example folder.

\begin{itemize}
  \item \lstinline{ping.c} send the request and wait for an acknowledgement.
  \item \lstinline{pong.c} send the acknowledgement.
\end{itemize}

\begin{figure}[H]
\centering

\begin{sequencediagram}

\newthread{A}{Mote A}{}
\newinst[1]{B}{Mote B}{}

\begin{call}[4]{A}{ping}{B}{pong}
\end{call}
\end{sequencediagram}

\caption{Description of the execution of the ping-pong example\label{fig:pingpongsequence}}

\end{figure}


