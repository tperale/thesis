\chapter{Time-Slotted Channel Hopping Implementation for LoRa\label{section:tsch}}

This chapter will cover the adaptation of the TSCH MAC protocol in Contiki OS to
work with LoRa.
It covers the issues induced by the current Contiki implementation and how I
managed to fix them to bring TSCH to LoRa.

The reader of this chapter will get a better understanding of the steps
required to port TSCH to a new LoRa module in Contiki.

\section{Timing parameters\label{section:timingparameters}}

The challenges of porting TSCH from 802.15.4 Phy consists in organizing
the strict timing needed to keep the notion of time slots.
In Contiki each part of the time slots are timed using modifiable
variable and the 802.15.4 Phy timing parameters are already
implemented in Contiki.
Porting TSCH to LoRa means to modify the timing to suits the LoRa communication
into the slotframe.
Because LoRa is way slower than 802.15.4 each parameter must increase.
This leads to a set of problems that we will cover in this section.

\paragraph{}

Time slot in TSCH is divided in four different timed parts used on
transmitting and receiving time slots as we can see in Fig~\ref{fig:tstiming}.

\begin{itemize}
  \item The setup time, made to prepare to transmit a packet.
    This timing is long enough for radio to be turned on and prepare the packet.
  \item The transmission or the time the packet is on air.
  \item The acknowledgement time, the time to setup the transmition of the
    acknowledgement if any.
    This delay is long enough for the receiver to receive the message,
    computes its check for correctness, and prepare the acknowledgement packet.
  \item The acknowledgement transmission time.
\end{itemize}

\begin{figure}[H]
  \centering

\begin{tikzpicture}[
  timeslot/.style={draw, rectangle, minimum size=1cm},
  description/.style={draw, rectangle, minimum size=1cm},
  arr/.style={help lines,black!70,<->},
]

\foreach [evaluate={\ts=int(mod(\i, 4))}] \i in {0,...,7} {
  \node (ts\i) [timeslot] at (\i, 0) {$TS_{\ts}$};
}
\node (ts8) [minimum height=1cm, minimum width=2cm, black!70] at (8.5, 0) {\ldots};

\draw[help lines, black!70]
  (ts8.north west) -- (ts8.north east) node[fill=white, black!70] {$\ldots$};
\draw[help lines, black!70]
  (ts8.south west) -- (ts8.south east) node[fill=white, black!70] {$\ldots$};

\draw[ultra thick] 
  (ts0.south west) rectangle (ts3.north east)
  (ts4.south west) rectangle (ts7.north east);

\draw[arr]
  ([yshift=10pt]ts0.north west) -- node[fill=white] {$SF0$} ([yshift=10pt]{ts3.north east});
\draw[arr]
  ([yshift=10pt]ts4.north west) -- node[fill=white] {$SF1$} ([yshift=10pt]{ts7.north east});
\draw[arr]
  ([yshift=10pt]ts8.north west) -- node[fill=white] {$SF\ldots$} ([yshift=10pt]{ts8.north east});

\begin{scope}[xshift=-1.2cm,yshift=-2cm,inner sep=0pt, outer sep=0pt]
  \node (desc) [fit={(-2.5,0) (0,1)}, label=center:{Transmitter}] {};
  \node (desc0) [description, fit={(0,0) (3,1)}, label=center:{tx\_offset}] {};
  \node (desc1) [description, fit={(3,0) (6,1)}, label=center:{max\_tx}] {};
  \node (desc2) [description, fit={(6,0) (9,1)}, label=center:{rx\_ack\_delay}] {};
  \node (desc3) [description, fit={(9,0) (12,1)}, label=center:{max\_ack}] {};
\end{scope}

\draw[help lines, black!70,-]
  ([yshift=0pt]ts4.south west) -- 
  ([yshift=0pt]{desc0.north west});
\draw[help lines, black!70,-]
  ([yshift=0pt]ts4.south east) -- 
  ([yshift=0pt]{desc3.north east});

\begin{scope}[xshift=-1.2cm,yshift=-3.5cm,inner sep=0pt, outer sep=0pt]
  \node (ddesc) [fit={(-2.5,0) (0,1)}, label=center:{Receiver}] {};
  \node (ddesc0) [description, fit={(0,0) (3,1)}, label=center:{rx\_offset}] {};
  \node (ddesc1) [description, fit={(3,0) (6,1)}, label=center:{rx\_wait}] {};
  \node (ddesc2) [description, fit={(6,0) (9,1)}, label=center:{tx\_ack\_delay}] {};
  \node (ddesc3) [description, fit={(9,0) (12,1)}, label=center:{max\_ack}] {};
\end{scope}

\draw[help lines, black!70,-]
  ([yshift=0pt]desc0.south west) -- 
  ([yshift=0pt]{ddesc0.north west});
\draw[help lines, black!70,-]
  ([yshift=0pt]desc3.south east) -- 
  ([yshift=0pt]{ddesc3.north east});

\end{tikzpicture}

\caption{Slotframes representation with the timings\label{fig:tstiming}}
\end{figure}


In its code base Contiki uses the following parameter, defined in
$\mu s$, to time each part of a time slot I just decribed.
For precise timing Contiki makes the distinction between the transmitter
and the receiver for each parameter.

\begin{lstlisting}
enum tsch_timeslot_timing_elements {
  tsch_ts_cca_offset,
  tsch_ts_cca,
  tsch_ts_tx_offset,
  tsch_ts_rx_offset,
  tsch_ts_rx_ack_delay,
  tsch_ts_tx_ack_delay,
  tsch_ts_rx_wait,
  tsch_ts_ack_wait,
  tsch_ts_rx_tx,
  tsch_ts_max_ack,
  tsch_ts_max_tx,
  tsch_ts_timeslot_length,
  tsch_ts_elements_count, /* Not a timing element */
};
\end{lstlisting}

\begin{itemize}
  \item \lstinline{tx_offset} represents the time needed for the packet to get
    prepared before getting transmitted on air. The end of
    \lstinline{tx_offset} is the moment the packet gets transmitted.
    The parameter can be the same as \lstinline{rx_offset} but using a lower value
    for the receiver allows clock drift between the receiver and the transmitter.
  \item \lstinline{max_tx} is the maximum transmission time (time-on-air) using
    LoRa. The parameter depends on the spreading factor and the maximum number
    of bytes transmittable.
  \item \lstinline{rx_ack_delay} buffering time between the moment the packet
    is transmitted and the acknowledgement transmission start.
    The parameter can the same as \lstinline{tx_ack_delay} but using a lower value
    for the receiver side allows a bigger clock drift.
  \item \lstinline{ts_max_ack} is the maximum transmission time (time on air) for
    the acknowledgement packet reception since its transmission started.
\end{itemize}

Among these parameters, I did not define \lstinline{tsch_ts_cca_offset} and
\lstinline{tsch_ts_cca} because \emph{CCA} is impossible since there is no way
for LoRa to detect ongoing transmission.
Also, \lstinline{tsch_ts_rx_tx} is not used anywhere in the Contiki codebase.

\paragraph{}

In this section I will introduce the change I made to the TSCH implementation
to allow bigger timing parameter that suits LoRa.

\subsection{Timing size issue}

The original implementation of TSCH in Contiki the size of the parameter
is not big enough to suit the long delay induced by using TSCH with LoRa.

Contiki stored each timing parameter in a 16-bit unsigned integer.

\begin{lstlisting}
typedef uint16_t tsch_timeslot_timing_usec[tsch_ts_elements_count];
\end{lstlisting}

This sets the maximum parameter value to $64436\mu s$ (ie. $64.4 ms$).
This value is not long enough to store LoRa transmissions at maximum payload.
It takes 379 ms to transmit a maximum payload packet with the fastest spreading
factor (SF 7).

I changed the TSCH implementation to use a 32-bit integer instead, allowing more
room for the parameters.

\begin{lstlisting}
typedef uint32_t tsch_timeslot_timing_usec[tsch_ts_elements_count];
\end{lstlisting}

This could be a breaking change for platform with very low memory available as
it will double the size of the timing array to 52 bytes.
I made sure to modify in my project each reference to
$tsch_timeslot_timing_usec$ to a $32 bit$ integer array but couldn't test the
platforms using those reference still worked properly.

\subsection{$\mu s$ and $ticks$ conversion}

To measure time embedded systems uses the number of processor instruction executed
per second (ticks/s), a constant value (even though its subject to small clock drift).
The change I made in the previous section allow the usage of bigger timing
parameters but the conversion from $\mu s$ to the number of ticks overflow
with conversion of over 131056 $\mu s$.

\paragraph{}

Contiki makes all the conversion with the following function.

\begin{lstlisting}
#define US_TO_RTIMERTICKS(US)  ((US) >= 0 ? \
 (((int32_t)(US)*(RTIMER_ARCH_SECOND) + 500000) / 1000000L) : \
 ((int32_t)(US)*(RTIMER_ARCH_SECOND) - 500000) / 1000000L)
\end{lstlisting}

A second in CC2538 equals to 32768 ticks, I can calculate the maximum time in $\mu s$
I can pass to the macro before overflowing.

\begin{equation}
  \label{eq:maxus}
  \begin{multlined}
  max(int32\_t) = \frac{2^{32}}{2} - 1 = 2147483647
  RTIMER\_ARCH\_SECOND = 32768 \\
  max(int32\_t) = x * RTIMER\_ARCH\_SECOND + 500000 \\
  x = ceil(\frac{max(int32\_t) - 500000}{RTIMER\_ARCH\_SECOND}) \\
  x = 131057 \\
  \end{multlined}
\end{equation}

Every conversion over 131056 $\mu s$ will make the 32 bits integer overflow.
This is why Contiki needs a new conversion function that uses a 64 bits integer
instead to avoid overflows.
I changed every conversion made with the previous version in TSCH
with this new one.

\begin{lstlisting}
#define US_TO_RTIMERTICKS_64(US) ((US) >= 0 ? \
 ((int32_t)(((int64_t)(US)*(RTIMER_ARCH_SECOND)+500000)/1000000L)) : \
 ((int32_t)((int64_t)(US)*(RTIMER_ARCH_SECOND)-500000)/1000000L))
\end{lstlisting}

\subsection{Interrupt Priority Clash}

The LoRa radio driver implemented in Chapter~\ref{section:radio}
uses a module that performs every radio operation using UART commands.
Each UART messages coming from the module trigger an interruption to receive the
message.
However the original implementation TSCH and its time slot are also run in
interrupt context.
This is done to avoid any interference when performing a transmission and to
use the real-time timer from the CC2538 (that get the best timing
precision) that trigger an interruption when the time is elapsed.
From a technical point of view,
the real-time timer interruption (\lstinline{SMT_IRQn}) has the same
priority as the UART1 interruption I need for the LoRa driver.
This is why I changed the \lstinline{rtimer_arch_schedule} implementation with
the following line to set a lower priority.

\begin{lstlisting}
NVIC_SetPriority(SMT_IRQn, 7);
\end{lstlisting}

The UART has now an higher priority than the time slot and will be executed even
during the time slot.

\subsection{Watchdog}

A watchdog is a timer used in embedded systems that must be reset in the
code at regular interval to verify the application has not crashed or is blocked
in an infinite loop. If the timer is not reset the system will reset.

\paragraph{}

The CC2538 watchdog is set to a period of 32,768 ticks or 1 second.
Interruption should be in theory as small as possible and that is why the watchdog timer
don't get reset during interrupt.
Because Contiki runs most TSCH operations inside an interrupt
context, the watchdog timer is not reset during the time slot.
During long waits or long transmissions, the watchdog is not reset and trigger
a system reset.
This is why I forced the timer reset when the program busy waits with the command
to also execute the \lstinline{watchdog_periodic} function, to reset the
watchdog timer.

% \subsection{Drift correction limitations}
%
% When acknowledging the reception of a packet, the receiver can bundle in the
% packet his drift from the transmitter in $\mu s$.
% This value is limited to a range of $-2047$ to $2047$ because of the structure
% of the packet that only give 12 bits to transmit this info.
%
% Bigger drift can occur in my implementation.
% I have to take into account this limitation to prevent overflows.

\section{Porting TSCH}

The previous section described the changes to be made to the current
implementation of TSCH.
This section will document all the parameter we need to implement at the driver
level.
Each MAC protocol have their own requirements the parameter for TSCH come from
the Contiki \href{wiki}{https://github.com/contiki-ng/contiki-ng/wiki/Documentation:-TSCH-and-6TiSCH}.

\subsection{Channels}

Channel hopping is one of the main features of TSCH, it makes the LoRa multi-hop
network reliable and resilient to external interference.

The ability to change the channel LoRa is transmitting from needs to be
implemented at the driver level.
For my experimentation, I only implemented three channels three of the eight
channels defined for LoRaWAN that can be selected with the \lstinline{RADIO_PARAM_CHANNEL}
argument in my driver.
During the formation of the TSCH network the joining node has to receive the beacon from
the coordinator by scanning random channel.
The other channels available in LoRaWAN could also be implemented but
using only three different channels reduces the time for a node to join
the TSCH network as having more channel clearly reduce the likelihood of
receiving the beacon with random scan.
Using three channels made the experimentation faster to execute while giving a realistic
representation of how channel hopping work.
This is the three different channel I defined

\begin{itemize}
  \item 868.1 MHz, 125 kHz BW
  \item 868.3 MHz, 125 kHz BW
  \item 868.5 MHz, 125 kHz BW
\end{itemize}

In a real-world deployement of TSCH with LoRa the eight channel of LoRaWAN
could be implemented (like in Fig~\ref{fig:channels}).
Adding more supported channels is trivial.

\subsection{Packet Duration\label{section:transmissiontime}}

To achieve power efficient LoRa multi-hop network, the radio must remains
off as long as possible.
One of the way to achieve that is to use precise timing for both the transmitting
and receiving timei slot so they stay in sync and do not have to wait for each
other with their radio on.

LoRa communication is dependant on the number of symbols transmitted
instead of the number of bytes as we saw in~\ref{section:toa}.
The current implementation in TSCH is not precise enough.

LoRa Transmission duration is known with the equation in \ref{eq:tpacket}.
The current implementation of Contiki calculates the time-on-air with the
\lstinline{TSCH_PACKET_DURATION(bytes)} macro.
This macro considers that the duration is linear as a function of the
number of bytes which is false in LoRa.
I changed the TSCH implementation to allow my own macro, and
I replaced it by the equation~\ref{eq:tpacket} for precise measurement
of the time on air.

\subsection{Radio delay}

With the same objective of power efficiency radio delays can be defined
to better predict when to turn on or off the radio.

Radio delays are hardware platform-dependent values, TSCH uses these three values

\begin{itemize}
  \item The transmission delay (\lstinline{RADIO_CONST_DELAY_BEFORE_TX})
    represent the delay from the module to get the message on air.
    When calling the function \lstinline{transmit} of the radio driver
    the transmission is not instantaneous and is dependant on the radio
    hardware.
  \item The reception delay (\lstinline{RADIO_CONST_DELAY_BEFORE_RX})
    represent the time for the module to be able to receive packet after the
    driver function \lstinline{on} is called.
  \item The delay before detect (\lstinline{RADIO_CONST_DELAY_BEFORE_DETECT})
    represent the time after the end of a packet transmission to be able to
    call the driver function \lstinline{read} to receive the packet.
\end{itemize}

Originally these parameters were constant but since they are dependant on the
radio hardware platform we are using with TSCH I had to change their value.
In the following sub-sections I describe how I found the value to use for these
parameters and how I made these parameter dependant on the number of byte
transmitted to achieve the most precise timing.

\subsubsection{Transmission delay}

The transmission delay is made of three parts

\begin{enumerate}
  \item UART message transmission to the module.
  \item Module message treatment (duration between the ends of $1.$ and the start of $3.$).
  \item Module sending back an acknowledgement.
\end{enumerate}

The program knows the message length and the acknowledgement length is constant.
The parameters $1.$ and $3.$ are then known with equation~\ref{eq:baudrate}.

\begin{equation}
  \label{eq:baudrate}
  UART(x) = \frac{x . (bit_{start} + bits_{data} + bit_{stop})}{baudrate}
\end{equation}

The parameter $2.$ is unknown, and the datasheet does not give any meaningful
information about the timing of the module.

For this benchmark, I did one set of measurement with nine messages of different
length using the shell of Appendix~\ref{appendix:shell}.
I measured the time between the end of the UART message transmission ($1.$) and the
start of the acknowledgement UART message ($3.$) with an oscilloscope.
The goal is to find a correlation between the number of bytes transmitted and
the treatment duration.

The poor resolution of the oscilloscope introduced some accuracy errors.
The longer measurements are already rounded in the ~100 $\mu s$ range.

\begin{figure}[H]
  \centering

  \begin{tikzpicture}

  \begin{groupplot}[group style={group size=1 by 2,
      horizontal sep=0pt,
      vertical sep=1cm},
      height=6cm,width=6cm,
  ]
  \nextgroupplot[
    width=0.8\textwidth,
    height=0.6\textwidth,
    axis x line=bottom,
    ymin=3000,
    ymax=24000,
    ylabel={$time_{\mu s}$},
    ylabel near ticks,
    yticklabel style={/pgf/number format/1000 sep=},
    axis y line=left,
    xmin=0,
    xmax=128,
    xlabel={bytes},
    xlabel near ticks
  ]
    \addplot[color=red,mark=x] coordinates {
      (1,3080)
      (4,3560)
      (8,4200)
      (21,6400)
      (32,8280)
      (58,12600)
      (64,13680)
      (86,17360)
      (124,23800)
    };
    \addlegendentry{Measurements}
    \addplot[color=blue] coordinates {
      (1,3021)
      (4,3528)
      (8,4204)
      (16,5556)
      (32,8260)
      (64,13668)
      (110,21442)
    };
    \addlegendentry{Predictions}

  \nextgroupplot[
    ycomb,
    width=0.8\textwidth,
    height=0.25\textwidth,
    axis lines=middle,
    ymin=-100,
    ymax=100,
    ylabel={$time_{\mu s}$},
    ylabel near ticks,
    yticklabel style={/pgf/number format/1000 sep=},
    xmin=0,
    xmax=128,
    xticklabels=\empty,
  ]
    \addplot[color=blue, mark=x] coordinates {
      (1,-59)
      (4,-32)
      (8,4)
      (16,6)
      (32,-20)
      (64,-12)
      (110,12)
    };

  \end{groupplot}

  \end{tikzpicture}

  \caption{Message treatment delay measurements and variation with the prediction\label{fig:transmissiondelay}}
\end{figure}

There is a clear linear relationship between the number of bytes transmitted and
the message treatment time.
Based on these measurements the best matching function is the following.

\begin{equation}
  \label{eq:transmissioncompute}
  T_{TX}(x) = 2852 + 169x \ \ \ \ (\mu s)
\end{equation}

The parameter returns the total transmission delay with the following
equation~\ref{eq:transmissiondelay}.

\begin{gather}
  \label{eq:transmissiondelay}
  msg_{length}(x) = x * 2 + length("radio~tx~\backslash r \backslash n") \\
  Delay_{TX}(x) = T_{TX}(x) + UART(msg_{length}(x)) + UART(length("ok \backslash r\backslash n"))
\end{gather}

The precision of this parameter is essential for synchronized communication
with TSCH.
It indicates to TSCH when to start its radio transmission precisely at the
end of \lstinline{TS_TX_OFFSET}.

\subsubsection{Reception delay}

It consists of the two following steps.

\begin{enumerate}
  \item Waking up the radio.
  \item Start listening to incoming messages.
\end{enumerate}

Because in my implementation the radio is always on (as I will explain in
Section~\ref{section:energyconsumption}) there is no real reception
delay.
This variable is then not used but should be taken into account in other
implementations.

\subsubsection{Detection delay\label{section:detectiondelay}}

% TODO Detection does not really represent that. It the delay between the start
% of the transmission and the acknowledgement by the platform since we can't
% really know when the communication start this value will be used out of
% context.
% Because RN2483 miss a way to detect transmission we are using variable in
% TSCH the wrong way.

This delay is made of two parts we can calculate when we know the message
length.

\begin{enumerate}
  \item Module message computation
  \item UART message transmission from the module to the platform
\end{enumerate}

I ran one set of 9 messages of different length to measure the delay between
the end of communication and the start of the
UART message transmission to the platform with an oscilloscope.

\begin{figure}[H]
  \centering
  \begin{tikzpicture}

  \begin{groupplot}[group style={group size=1 by 2,
      horizontal sep=0pt,
      vertical sep=1cm},
      height=6cm,width=6cm,
  ]
  \nextgroupplot[
    width=0.8\textwidth,
    height=0.6\textwidth,
    axis x line=bottom,
    ymin=0,
    ymax=55000,
    ylabel={$time_{\mu s}$},
    ylabel near ticks,
    yticklabel style={/pgf/number format/1000 sep=},
    axis y line=left,
    xmin=0,
    xmax=128,
    xlabel={bytes},
    xlabel near ticks
  ]
    \addplot[color=red,mark=x] coordinates {
      (1,820)
      (2,1420)
      (3,1940)
      (4,2350)
      (8,4240)
      (16,8160)
      (32,15600)
      (64,31100)
      (110,53200)
    };
    \addlegendentry{Measurements}
    \addplot[color=blue] coordinates {
      (1,898)
      (4,2332)
      (8,4244)
      (16,8068)
      (32,15716)
      (64,31012)
      (110,53000)
    };
    \addlegendentry{Predictions}

  \nextgroupplot[
    ycomb,
    width=0.8\textwidth,
    height=0.25\textwidth,
    axis lines=middle,
    ymin=-200,
    ymax=200,
    ylabel={$time_{\mu s}$},
    ylabel near ticks,
    yticklabel style={/pgf/number format/1000 sep=},
    xmin=0,
    xmax=128,
    xticklabels=\empty,
  ]
    \addplot[color=blue, mark=x] coordinates {
      (1,68)
      (4,-18)
      (8,4)
      (16,-92)
      (32,116)
      (64,-88)
      (110,-200)
    };

  \end{groupplot}

  \end{tikzpicture}
  \caption{Reception delay measurements and variation with the prediction\label{fig:receptiondelay}}
\end{figure}


There is also a clear linear relationship between the number of bytes transmitted
and the reception treatment time.
The best matching function is the following based on the measurements.

\begin{equation}
  \label{eq:detectcomp}
  T_{detect}(x) = 408x + 49 . N_{sym}(x) \ \ \ (\mu s)
\end{equation}

The total detection delay is calculated with the following formula.

\begin{gather}
  msg_{length}(x) = x * 2 + length("radio~rx~~\backslash r \backslash n") \\
  \label{eq:detectdelay}
  Delay_{detect}(x) = T_{detect}(x) + UART(msg_{length}(x))
\end{gather}

\subsection{Packet Timestamp}

As every participant of the LoRa multi-hop network is subject to clock drift,
to keep the power usage of the network as low as possible it is
crucial to keep the time slots of each nodes synchronized.
Coordinator nodes are used as reference to synchronize the clock of the nodes
receiving packet from it by including the timestamp in the packet header.
As we saw in Figure~\ref{fig:sync} each time slot uses the same offset before
starting the transmission and knowing the packet timestamp precisely allows
correct its drift.

A feature that the RN2483 module lacks of is a way to know when the reception started
precisely.
There is only a timestamp saved when the packet is fully received on the platform.
I calculate the real timestamp (the communication start) with the
\emph{detection delay} covered in section~\ref{section:detectiondelay} and
equation~\ref{eq:detectdelay}, as well as the transmission time of
section~\ref{section:transmissiontime} and equation~\ref{eq:tpacket}.
The implementation is available in appendix~\ref{code:timestampimpl}.

By knowing the start of a transmission the network can keep running synchronized
and thus achieve low power consumption.

% \subsection{Mitigating the drift}

% The transmitter can know in advance the moment the receiver will fully
% receive the message. In the previous implementation of TSCH the transmitter
% slot operation used to wait a constant time between transmissions. I modified
% this to include the detection delay at the transmitter side and the receiver
% will wait a constant time at reception instead.

% This will mitigate the effect of drifting timeslots, big messages are increasing
% the detection time that could lead to miss the time frame for sending the
% acknowledgement.

\section{Drift issues\label{section:driftissues}}

During the adaptation of the TSCH protocol for the RN2483, I stumbled upon huge
clock drifts that made me unable to run a reliable network using TSCH.

After several improvement and research, I can conclude that the RN2483
module used in this work and the formula I came up in the previous section its
delays are still too imprecise to run with the Contiki TSCH
implementation out of the box.

The functions \ref{eq:transmissiondelay} and \ref{eq:detectcomp} can
create imprecisions of multiple ms.
The original Contiki implementation statistically corrected the internal clock
drift of the motes with the function
\lstinline{tsch_timesync_adaptive_compensate}.
The function learned from the nodes drift from the time source to predict the
difference.

Other clock synchronization method could have been investigated but they would
still rely on non-appropriate hardware.
The actual solution for this drift issue is to base the implementation on radio
driver that has sufficient documentation to provide precise timing informmation
on the delays.

Figure~\ref{fig:driftmeasurement} shows how
\lstinline{tsch_timesync_adaptive_compensate} introduces a lot more drift on the
receiver time slot on a network of two nodes running for 10 minutes.
The lower overall drift explains why I stopped using this function to conduct my experiments.
Re-using the adaptive compensation should be done with a module that can
provide an acknowledgement at packet reception and precise timing from the
manufacturer.

\begin{figure}[H]
  \centering
  \begin{tikzpicture}
    \begin{axis}
      [
      width=\textwidth,
      height=0.5\textwidth,
      ytick={1,2},
      yticklabels={w/o, w},
      xlabel={$time_{ms}$},
      xlabel near ticks,
      ]
      \addplot+[
      boxplot prepared={
        median=0.25,
        upper quartile=0.40,
        lower quartile=-1.95,
        upper whisker=7.9,
        lower whisker=-8.9
      },
      ] coordinates {};
      \addplot+[
      boxplot prepared={
        median=3.33,
        upper quartile=6.500,
        lower quartile=-4.100,
        upper whisker=13.333,
        lower whisker=-16.666
      },
      ] coordinates {};
    \end{axis}
  \end{tikzpicture}
  \caption{Drift measurement with and without \lstinline{tsch_timesync_adaptive_compensate}\label{fig:driftmeasurement}}
\end{figure}

\section{Validation\label{section:tschtesting}}

To review the implementation and adaptation I made during this section, I will
run experiments that will validate the aspect I wanted to develop for my
LoRa multi-hop network.

The first one is enhancing the range of LoRa by using multi-hop routing.
This is done in section~\ref{section:tschrpl} where I run a custom schedule
to force the nodes to use multiple hops to reach its destination.

The second aspect is achieving a power efficient network demonstrated in
section~\ref{section:tschrpl} and \ref{section:tschorchestra}.
This is done by sending messages over multiple hops and thus showing the time
slots are synchronized between each nodes by reaching its destination.

The last aspect is the reliability of the multi-hop network that will be
demonstrated in section~\ref{section:jamming}. I will demonstrate the resilience
of LoRa multi-hop network with TSCH to external interference by using a jamming
device to mimic a busy channel.

\subsection{Energy consumption\label{section:energyconsumption}}

Because of the design of the RN2483 which gives priority to the ease of use,
a proper measurement of the energy consumption where we measure the radio
usage is not relevant because the module is wasting energy in two places.

\begin{itemize}
  \item The slow UART communication between the platform and the module. For
    instance a max payload transmission to the module takes $86 ms$.
    That is $86 ms$ of wasted energy with the module powered on. The same goes
    for every other command used by my driver.
  \item Because we can not receive feedback from the module when it detects a
    preamble each timeslot must wait for the maximum transmission time to be
    sure there is no ongoing transmission during the timeslot.
\end{itemize}

% TODO There is also a third place: the huge tx_offset i set intentionaly to
% mitigate the effect of the drift from the module. It's a lot of waster energy
% that could be avoided.

The first issue could be solved with a module that uses a faster communication
protocol like SPI and that does not focus on human-readable command format.
The second can be solved by using a module that include a way to acknowledge
receptions.

From the result of the validation experiment I will describe in the next section
we can conclude that it is possible to use TSCH to create a power efficient LoRa
multi-hop network but not with the RN2483.
We won't be able to compare the power numbers to other existing solution
because of this hardware issue.

% Because I knew I would not be able to measure meaningful data from my
% implementation concerning energy consumption, I did not take time to experiment
% with the sleep feature of the module and did not proceed to do further energy
% consumption analysis.

\subsection{Testing setup}

I used four Zolertia RE-Mote Rev-B platform connected to RN2483 breakout board.
The MAC addresses of the nodes I used are the following

\begin{enumerate}
  \item $:2f2a$
  \item $:2f09$
  \item $:2dac$
  \item $:2dbc$
\end{enumerate}

\subsection{RPL Custom Schedule\label{section:tschrpl}}

For this example, I will create a custom schedule and demonstrate
my implementation can achieve multi-hop routing to achieve longer range
transmission.
I will also demonstrate participants of the network are synchronized which shows
that power efficient multi-hop network is achieved.

\paragraph{}

I will run a UDP client and server on each node and
send a message between two nodes to see the packet get routed correctly with
RPL and the multi-hop topology.

Custom schedules in Contiki can be done from anywhere with the help of the
function \lstinline{tsch_schedule_add_link}, the code for the custom schedule
is available in the Appendix~\ref{code:customsched}.
The schedule I made is schematized in Figure~\ref{fig:customsched}, the $:2f2a$
node is the network coordinator.

\begin{figure}[H]
\centering
  \begin{tikzpicture}[]

  \begin{scope}[
      xshift=-0.2cm,
      asn/.style={black!70, minimum width=2cm},
      timeslot/.style={draw, rectangle, minimum width=2cm, minimum height=1cm},
      arr/.style={help lines,black!70,<->},
  ]
    \foreach \i in {0,...,7} {
      \node (ts\i) [asn] at (2*\i, 1) {$TS_{\i}$};
    }
    \node (tss0) [timeslot] at (0, 0) {\tiny Shared};
    \node (tss1) [timeslot] at (2, 0) {\tiny :2dac $\rightarrow$ :2f2a};
    \node (tss2) [timeslot] at (4, 0) {\tiny :2dbc $\rightarrow$ :2f2a};
    \node (tss3) [timeslot] at (6, 0) {\tiny :2dbc $\rightarrow$ :2dac};
    \node (tss4) [timeslot] at (8, 0) {\tiny :2f2a $\rightarrow$ :2dbc};
    \node (tss5) [timeslot] at (10, 0) {};
    \node (tss6) [timeslot] at (12, 0) {};
    \node (tss7) [timeslot] at (14, 0) {};
  \end{scope}
  \begin{scope}[xshift=5cm,yshift=-2cm,->,>=stealth',shorten >=1pt,auto,node distance=2.5cm]
    \tikzstyle{every state}=[thick,draw=gray!50,fill=gray!20,draw=none,text=black]

    \node[state,thick,draw=red!40]         (A) [] {:2f2a};
    \node[state]         (B) [below of=A]       {:2dac};
    \node[state]         (C) [right of=B]       {:2dbc};

    \path (A) edge [bend left]    node {$TS_4$} (C)
          (B) edge [bend left]     node {$TS_1$} (A)
          (C) edge [bend left]     node[above right] {$TS_2$} (A)
              edge [bend left] node {$TS_3$} (B);
  \end{scope}
\end{tikzpicture}
\caption{Custom Schedule\label{fig:cmdstate}}
\end{figure}


I used the NETSTACK in Figure~\ref{fig:tschstack} for this example.

\begin{figure}[H]
\centering
  \begin{tikzpicture}[->,>=stealth',shorten >=1pt,auto,node distance=1.6cm]

  \tikzstyle{comment}=[
    right=2pt,
    font=\small,
    fill=white,
    text=black,
    draw=black,
  ]

  \tikzstyle{every state}=[rectangle,thick,
    draw=black,fill=gray!20,text=black,
    minimum width= 6cm,
    minimum height= 1.20cm
  ]

  \tikzstyle{smallstate}=[rectangle,thick,
    draw=black,fill=gray!10,text=black,
    minimum width= 4cm,
    minimum height= 1.20cm
  ]

  \node[smallstate]         (A)                    {IPv6};
  \node[smallstate]         (B) [below of=A]       {RPL Lite};
\begin{scope}[on background layer]
  \node[state, fit=(A)(B)] (AB)                 {};
\end{scope}
  \node[state,below=1cm]         (C) [below of=AB]       {TSCH};
  \node[state]         (D) [below of=C]       {LoRa};

  \node[comment]       at (AB.north west) {Network Layer};
  \node[comment]       at (C.north west) {MAC Layer};
  \node[comment]       at (D.north west) {Physical Layer};
  % \path (A) edge [bend right]    node {     } (B)
  %           edge [bend left]     node {     } (C)
  %       (B) edge [bend left]     node {     } (C)
  %           edge                 node {     } (D)
  %       (C) edge [bend left]     node {     } (A)
  %       (D) edge [bend right=85] node {     } (A);
\end{tikzpicture}
\caption{NETSTACK for the TSCH project\label{fig:tschstack}}
\end{figure}


I will send a UDP message from $:2f09$ to $:2dbc$ to show the multi-hop routing
feature.
The message coming from $:2f09$ will have to go through $:2dac$ and $:2f2a$ to
reach $:2dbc$ (see Fig~\ref{fig:route}).

\begin{figure}[H]
\centering
  \begin{tikzpicture}[]

  \begin{scope}[xshift=5cm,yshift=-2cm,->,>=stealth',shorten >=1pt,auto,node distance=2.5cm]
    \tikzstyle{every state}=[thick,draw=gray!50,fill=gray!20,draw=none,text=black]

    \node[state]         (A) [] {:2f2a};
    \node[state]         (B) [below left of=A]       {:2dac};
    \node[state]         (C) [below right of=A]       {:2dbc};
    \node[state]         (D) [below left of=B]       {:2f09};

    \path (A) edge  node {$3$} (C)
          (B) edge  node {$2$} (A)
          (D) edge  node {$1$} (B);
  \end{scope}
\end{tikzpicture}
\caption{Route to send a message from :2f09 to :2dbc\label{fig:route}}
\end{figure}


% TODO talk about network formation time to join it. Graph ?

The $:2f2a$ node is set as the network coordinator, it will be the first node to be
run in the network and the other nodes will join him.
To send a message to another node from $:2f09$ we must first wait it
knows the root node to send the packet to.
We can check this with the command \lstinline{routes}.

\begin{lstlisting}[language=none]
#0012.4b00.14d5.2f09> routes
Default route:
-- fe80::212:4b00:14d5:2f2a (lifetime: infinite)
\end{lstlisting}

The root node should also reach the end node.

\begin{lstlisting}[language=none]
#0012.4b00.14d5.2f2a> ip-nbr
Node IPv6 neighbors:
-- fe80::212:4b00:14d5:2dbc, router 0, state Reachable
-- fe80::212:4b00:14d5:2dac, router 0, state Reachable
-- fe80::212:4b00:14d5:2f09, router 0, state Reachable
\end{lstlisting}

To send a 'ping' message to another node, I made a custom shell
command \lstinline{msg_to} that sends a UDP message to the IPv6 address
passed in argument.
Each node of the network is running a UDP server on port 8765.

\begin{lstlisting}[language=none]
#0012.4b00.14d5.2f09> msg_to 0012.4b00.14d5.2dbc
Sending message to fd00::212:4b00:14d5:2dbc
[INFO:TSCH] send packet to 0012.4b00.14d5.2f2a with seqno 46, queue 7 8, len 21 51
[INFO:TSCH] {asn 00.00001eca link  0   7   0  0  0 ch  1} uc-1-0 tx LL-2f09->LL-2f2a, len  51, seq  46, st 2  1
\end{lstlisting}

The \lstinline{:2f2a} node receives the following message on the reception of the
request.

\begin{lstlisting}[language=none]
[INFO:TSCH] received from 0012.4b00.14d5.2f2a with seqno 155
[INFO:MAIN] Received request 'ping' from fd00::212:4b00:14d5:2f09
\end{lstlisting}

This shows LoRa multi-hop network using TSCH can send and receive UDP messages
using RPL to route the message through multiple hops and thus improving the
range of LoRa by using multi-hop routing.
Further research needs to be conducted to test this example in a real-world
setup by spreading the motes hundreds of meters apart.
The context of this thesis made it impossible to do such a real-world test.

\paragraph{}

The second part of this test is to demonstrate the node keep synchronized
through time and thus prooving this solution is power efficient.
I used the same setup where I sent 50 packets between two nodes at a 1 minute
interval. I counted the number of time the motes gets desynchronized from the
network as well as the number of retransmission performed to reach the
destination that could show the potential desynchronized slots.

\subsection{Orchestra\label{section:tschorchestra}}

Besides validating that LoRa and TSCH can support a IPv6 transport
layer, I conducted experiments to test that other schedulers can be used on top
of TSCH. More concretely, I use Orchestra autonomous scheduler on top of my TSCH network.
Transmitting a message between two nodes by the mean of a hop would again show
the TSCH implementation for LoRa is compatible with this network stack.

\paragraph{}

I ran a three nodes network made of the following addresses.

\begin{enumerate}
  \item $:2dac$ (coordinator)
  \item $:2f2a$
  \item $:2dbc$
\end{enumerate}

Orchestra is simply enabled on Contiki by adding the next line in the project
\lstinline{Makefile}.

\begin{lstlisting}
MODULES += (CONTIKI_NG_SERVICES_DIR)/orchestra
\end{lstlisting}

The network formation and route building take more time to be created than with
the RPL custom schedule.
But after giving the network some time to organize itself, the coordinator gets
the full knowledge of all the routes available in the network.

\begin{lstlisting}
Routing links (3 in total):
-- fd00::212:4b00:14d5:2dac  (DODAG root) (lifetime: infinite)
-- fd00::212:4b00:14d5:2dbc  to fd00::212:4b00:14d5:2dac
-- fd00::212:4b00:14d5:2f2a  to fd00::212:4b00:14d5:2dbc
\end{lstlisting}

This means the DODAG in Figure~\ref{fig:orchestra} is created by using Orchestra.

\begin{figure}[H]
\centering
  \begin{tikzpicture}[]

  \begin{scope}[xshift=5cm,yshift=-2cm,->,>=stealth',shorten >=1pt,auto,node distance=2.5cm]
    \tikzstyle{every state}=[thick,draw=gray!50,fill=gray!20,draw=none,text=black]

    \node[state]         (A) [] {:2f2a};
    \node[state]         (B) [right of=A]       {:2dbc};
    \node[state]         (C) [right of=B]       {:2dac};

    \path (A) edge  node {$1$} (B)
          (B) edge  node {$2$} (C);
  \end{scope}
\end{tikzpicture}
\caption{Route to send a message from :2f2a to :2dac with Orchestra\label{fig:orchestra}}
\end{figure}


The node $:2f2a$ has no direct route to $:2dac$.
To prove the multi-hop routing is also correctly working with Orchestra, I will
send a message between those two nodes.

\paragraph{}

Sending a UDP message to $:2dac$ worked seamlessly by passing by the
intermediate node $:2dbc$.
This shows the TSCH adaptation for LoRa can work with a scheduler specifically
made for TSCH and RPL prooving the characteristics of TSCH were not lost when
adapting the protocol.

\subsection{Channel Jamming\label{section:jamming}}

This is inspired by the experiment conducted in~\cite{tschoverlora}.
The goal is to demonstrate the resilience of transmitting messages with LoRa and
TSCH over a busy channel.
To conduct this experiment, I used two motes running a TSCH network and the third one
as a jammer like in Figure~\ref{fig:jammer}.

\begin{figure}[H]
\centering

\begin{tikzpicture}[auto, node distance=2cm,>=latex']

\tikzset{
  block/.style = {draw, fill=white, rectangle, minimum height=3em, minimum width=2cm},
  input/.style = {coordinate},
  output/.style = {coordinate},
  pinstyle/.style = {pin edge={to-,t,black}},
  radiation/.style={decorate,decoration={expanding waves,angle=12,segment length=4pt}},
  zigzag/.style={to path={ -- ($(\tikztostart)!.55!-7:(\tikztotarget)$) -- ($(\tikztostart)!.45!7:(\tikztotarget)$) -- (\tikztotarget) \tikztonodes}}
}
\node[block](tx){Transmitter Node};
\node[block,above right= 2cm of tx](ttx){Jammer};
\node[block,right = 5cm of tx](rx){Coordinator Node};

\draw[radiation] ([shift={(1cm,0cm)}]tx.east)-- node [above=5mm] {} ([shift={(-1cm,0cm)}]rx.west);
\draw[-stealth,line width=1mm] (ttx.south) to[zigzag] +(0,-1.5);
\end{tikzpicture}

\caption{Structure of the channel jamming test\label{fig:jammer}}
\end{figure}


One mote from the TSCH network will send UDP messages to the coordinator
every 30 seconds.
Meanwhile, the third mote constantly sends messages on the same channel to
simulate a busy channel of the network.
This experiment shows how the protocol can deal with interference by retransmitting
the message on a different channel.

\paragraph{}

To achieve this experiment I ran a benchmark where I sent fifty differents UDP packets
with a 1 minute interval and counted the number of retransmission needed at
the transmitter side to achieve a transmission.
Throughout the benchmark the $Channel_0$ was coninuously jammed.

The 50 packets I sent to the coordinator in the course of my experiments
all managed to reach the destination.
Figure~\ref{fig:retransmission} shows the amount of retransmission for each
packet transmitted during the experiment.
Further research needs to be conducted to quantify the effect of channel
jamming on latency. However it requires a digital analyser which was not
accessible in the context of this thesis

\begin{figure}[H]
  \centering
  \begin{tikzpicture}

  \begin{groupplot}[group style={group size=1 by 2,
      horizontal sep=0pt,
      vertical sep=1cm},
      height=6cm,width=6cm,
  ]
  \nextgroupplot[
    ycomb,
    width=0.8\textwidth,
    height=0.25\textwidth,
    axis lines=middle,
    ymin=0,
    ymax=4,
    ylabel={$Retry_{number}$},
    ylabel near ticks,
    yticklabel style={/pgf/number format/1000 sep=},
    xmin=0,
    xmax=50,
    xlabel={$Transmission_{number}$},
    xlabel near ticks,
  ]
    \addplot[color=blue, mark=x] coordinates {
      (1,1)
      (2,2)
      (3,0)
      (4,0)
      (5,0)
      (6,1)
      (7,0)
      (8,0)
      (9,1)
      (10,0)
      (11,1)
      (12,1)
      (13,2)
      (14,0)
      (15,2)
      (16,0)
      (17,1)
      (18,1)
      (19,1)
      (20,0)
      (21,4)
      (22,0)
      (23,0)
      (24,1)
      (25,0)
      (26,0)
      (27,2)
      (28,1)
      (29,0)
      (30,0)
      (31,1)
      (32,1)
      (33,0)
      (34,0)
      (35,0)
      (36,1)
      (37,2)
      (38,0)
      (39,0)
      (40,0)
      (41,1)
      (42,1)
      (43,0)
      (44,0)
      (45,3)
      (46,1)
      (47,0)
      (48,0)
      (49,1)
      (50,1)
    };
  \end{groupplot}
  \end{tikzpicture}
  \caption{Number of retransmission depending on the packet send during the experiment\label{fig:retransmission}}
\end{figure}

This experiment shows the resilience of my LoRa multi-hop network to external
interference or congestionned radio channel with the help of the channel hopping
mechanism from TSCH.

\section{Conclusion}

This chapter covered the main elements to bring TSCH for the LoRa communication
protocol.
The concept and problems I have stumbled upon during my implementations for the
RN2483 can be used for any LoRa radio driver implementation.
The two main elements to take into account when porting TSCH are the TSCH timing
parameters and the driver's parameters.
They were both covered in their own section where I detailed the steps I went
through giving the reader the key to understand what these parameters do.
I demonstrated that channel hopping is working by jamming one of
the channel.
In another example, I tested the multi-hop routing capability of my
implementation by running respectively RPL and Orchestra on top of TSCH and
LoRa.

\paragraph{}

I could not achieve optimal timing parameters and did not detailed the ones
I used because of the clock drift I encountered in my
implementation.
I attributed this issue to the lack of precise timestamp when the reception of
a communication start.
Also, because of the inherent issues of the RN2483 module, I did not conduct
a power analysis on my work.
