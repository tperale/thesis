\chapter{Table}

\section{LoRa: Bit rate and data rate}

\begin{table}[h!]
\centering
\begin{tabular}{@{}lll@{}}
\toprule
BW (kHz) & SF & bit rate (bits/sec) \\ \midrule
125     & 7 & 6836 \\
125     & 8 & 3906 \\
125     & 9 & 2197 \\
125     & 10 & 1220 \\
125     & 11 & 671 \\
125     & 12 & 366 \\
250     & 7 & 13671 \\
250     & 8 & 7812 \\
250     & 9 & 4394 \\
250     & 10 & 2441 \\
250     & 11 & 1342 \\
250     & 12 & 732 \\ \bottomrule
\end{tabular}
\caption{Bit rates depending on BW, and SF\label{table:bitrate}}
\end{table}

\begin{table}[h!]
\centering
\begin{tabular}{@{}lllll@{}}
\toprule
CR & BW (kHz) & SF7 (bits/sec) & SF8 (bits/sec) & SF10 (bits/sec) \\ \midrule
\multirow{2}{*}{$\frac{4}{5}$} & 125000   & 5469           & 3125           & 976             \\
  & 250000   & 10937          & 6250           & 1953            \\
\multirow{2}{*}{$\frac{4}{6}$} & 125000   & 4557           & 2604           & 813             \\
  & 250000   & 9114           & 5208           & 1627            \\
\multirow{2}{*}{$\frac{4}{8}$} & 125000   & 3417           & 1954           & 660             \\
  & 250000   & 6835           & 3906           & 1220            \\ \bottomrule
\end{tabular}
\caption{Data rates depending on CR, BW, and SF\label{table:datarate}}
\end{table}

\section{RN2483 Timing Measurements and Predictions}

\begin{table}[H]
\centering
\begin{tabular}{@{}lll@{}}
\toprule
\multicolumn{1}{|c|}{\cellcolor[HTML]{C0C0C0}Bytes} & Time ($\mu s$) & Prediction ($\mu s$) \\ \midrule
1                                                   & 3080      & 3021       \\
4                                                   & 3560      & 3528       \\
8                                                   & 4200      & 4204       \\
16                                                  & 5550      & 5556       \\
32                                                  & 8280      & 8260       \\
64                                                  & 13680     & 13668      \\
110                                                 & 21430     & 21442      \\ \bottomrule
\end{tabular}
\caption{Transmission Time\label{table:measurementtx}}
\end{table}

\begin{table}[H]
\centering
\begin{tabular}{@{}lll@{}}
\toprule
\multicolumn{1}{|c|}{\cellcolor[HTML]{C0C0C0}Bytes} & Time ($\mu s$) & Prediction ($\mu s$) \\ \midrule
1                                                   & 820       & 898        \\
4                                                   & 2350      & 2332       \\
8                                                   & 4240      & 4244       \\
16                                                  & 8160      & 8068       \\
32                                                  & 15600     & 15716      \\
64                                                  & 31100     & 31012      \\
110                                                 & 53200     & 53000      \\ \bottomrule
\end{tabular}
\caption{Reception Spreading Factor 7\label{table:rxsf7}}
\end{table}

\begin{table}[H]
\centering
\begin{tabular}{@{}lll@{}}
\toprule
\multicolumn{1}{|c|}{\cellcolor[HTML]{C0C0C0}Bytes} & Time ($\mu s$) & Prediction ($\mu s$) \\ \midrule
1                                                   & 1500      & 1529       \\
4                                                   & 3240      & 2963       \\
8                                                   & 5060      & 4875       \\
16                                                  & 8840      & 8699       \\
32                                                  & 16700     & 16347      \\
64                                                  & 31700     & 31643      \\
110                                                 & 53300     & 53631      \\ \bottomrule
\end{tabular}
\caption{Reception Spreading Factor 8\label{table:rxsf8}}
\end{table}

\begin{table}[H]
\centering
\begin{tabular}{@{}lll@{}}
\toprule
\multicolumn{1}{|c|}{\cellcolor[HTML]{C0C0C0}Bytes} & Time ($\mu s$) & Prediction ($\mu s$) \\ \midrule
1                                                   & 3160      & 3287       \\
4                                                   & 4560      & 4721       \\
8                                                   & 6500      & 6633       \\
16                                                  & 10120     & 10457      \\
32                                                  & 17900     & 18105      \\
64                                                  & 33200     & 33401      \\
110                                                 & 55200     & 55389      \\ \bottomrule
\end{tabular}
\caption{Reception Spreading Factor 9\label{table:rxsf9}}
\end{table}

\begin{table}[H]
\centering
\begin{tabular}{@{}lll@{}}
\toprule
\multicolumn{1}{|c|}{\cellcolor[HTML]{C0C0C0}Bytes} & Time ($\mu s$) & Prediction ($\mu s$) \\ \midrule
1                                                   & 6100      & 6235       \\
4                                                   & 7400      & 7669       \\
8                                                   & 9660      & 9581       \\
16                                                  & 13560     & 13405      \\
32                                                  & 21040     & 21053      \\
64                                                  & 36400     & 36349      \\
110                                                 & 58400     & 58337      \\ \bottomrule
\end{tabular}
\caption{Reception Spreading Factor 10\label{table:rxsf10}}
\end{table}

\chapter{Code}

\section{Driver implementation structure\label{appendix:driver}}

My implementation structure took inspiration from the work
in~\cite{8847137} and the \emph{RIOT-OS}
project\footnote{\url{https://github.com/RIOT-OS/RIOT}}.

Following is a listing of the different files I created and their
usage. % \improvement{I don't like this phrase}

\begin{description}
  \item[lora] LoRa radio specific functions declaration, and configuration.
    This includes functions used to calculate the time-on-air and
    LoRa PHY configuration structure and parameters.
  \item[rn2483-include] contains the declaration both needed by
    \lstinline{rn2483-uart} and \lstinline{rn2483} to avoid cycling inclusion
    that mess with the compiler.
  \item[rn2483-uart] Lower level implementation of the function to directly
    communicate with the module via UART\@.
    It contains hardware-specific implementation for the UART\@.
  \item[rn2483-api] Complete coverage of the RN2483 commands. Using functions
    from this file allows the user of the driver to not write
    each command by hand and use standard structure instead.
  \item[rn2483] Implementation of the Contiki radio device driver.
    It contains the standardized way of Contiki to use and configure radio
    devices.
    All the radio drivers in Contiki share this structure.
\end{description}

\subsection{The RN2483 Shell\label{appendix:shell}}

The \emph{RN2483 shell} available in
\lstinline{/example/platform-specific/lora/rn2483}
is a tool to interact directly with the module.

This tool extends the existing shell available in Contiki by writing a set
of commands to test and interact with the RN2483 via UART\@.
It was used during the development of the driver as a way to test parameters
as well as the synchronous communication implementation.

Here is the list of the added commands to the Contiki shell.

\begin{lstlisting}[language=none]
'> sys': Send a 'system' command to the RN2483 Module.
'> mac': Send a 'mac' command to the RN2483 Module.
'> radio': Send a 'radio' command to the RN2483 Module.
'> tx': Transmit a radio message with the RN2483.
'> sleep': Put RN2483 into sleep mode.
'> wakeup': Force wakeup of RN2483.
'> rn2483_reset': Hardware Reset of the RN2483 Module.
'> help': Shows this help.
\end{lstlisting}

Each project is bundled with a \lstinline{project-conf.h}
configuration file in which we can define the parameters of our project.
To set my LoRa driver as the physical layer implementation we must add the
following line to change the NETSTACK configuration.

\begin{lstlisting}
#define NETSTACK_CONF_RADIO rn2483_radio_driver
\end{lstlisting}

To compile and upload the project go to the
\lstinline{/example/platform-specific/lora/rn2483}
folder and then run.

\begin{lstlisting}
make TARGET=zoul BOARD=remote-revb PORT=/dev/ttyUSBx test-rn2483.upload
\end{lstlisting}

And run the following command to access the shell.

\begin{lstlisting}
make TARGET=zoul BOARD=remote-revb PORT=/dev/ttyUSBx login
\end{lstlisting}


\section{Packet timestamp implementation\label{code:timestampimpl}}

\begin{lstlisting}[language=c]
static radio_result_t
rn2483_get_object(radio_param_t param, void *dest, size_t size)
{
  switch(param) {
  case RADIO_PARAM_LAST_PACKET_TIMESTAMP:
    if(size != sizeof(rtimer_clock_t) || !dest) {
      return RADIO_RESULT_INVALID_VALUE;
    }
    *(rtimer_clock_t *)dest = RN2483_DEV.packet_timestamp
      - US_TO_RTIMERTICKS_64(
          RX_RECEIVING_DELAY_US(RN2483_DEV.rx_len)
          + t_packet(&(RN2483_DEV.radio), RN2483_DEV.rx_len)
          + TRANSMISSION_TIME_US(RX_CMD_LENGTH(RN2483_DEV.rx_len))
      );
    return RADIO_RESULT_OK;
  }
}
\end{lstlisting}



\section{Custom Schedule Code\label{code:customsched}}

\begin{lstlisting}
static linkaddr_t node_1_address = {{
  0x00, 0x12, 0x4b, 0x00, 0x14, 0xd5, 0x2f, 0x2a
}};
static linkaddr_t node_2_address = {{
  0x00, 0x12, 0x4b, 0x00, 0x14, 0xd5, 0x2d, 0xac
}};
static linkaddr_t node_3_address = {{
  0x00, 0x12, 0x4b, 0x00, 0x14, 0xd5, 0x2d, 0xbc
}};
static linkaddr_t node_4_address = {{
  0x00, 0x12, 0x4b, 0x00, 0x14, 0xd5, 0x2f, 0x09
}};

void
tsch_schedule_custom(void)
{
  struct tsch_slotframe *sf_custom;
  tsch_schedule_remove_all_slotframes();
  sf_custom = tsch_schedule_add_slotframe(0, TSCH_SCHEDULE_DEFAULT_LENGTH);
  tsch_schedule_add_link(sf_custom,
      LINK_OPTION_RX | LINK_OPTION_TX | LINK_OPTION_SHARED | LINK_OPTION_TIME_KEEPING,
      LINK_TYPE_ADVERTISING, &tsch_broadcast_address,
      0, 0);

  if (linkaddr_node_addr.u8[7] == node_1_address.u8[7]) {
    tsch_schedule_add_link(sf_custom, LINK_OPTION_RX, LINK_TYPE_NORMAL, &node_2_address, 1, 0);
    tsch_schedule_add_link(sf_custom, LINK_OPTION_RX, LINK_TYPE_NORMAL, &node_3_address, 2, 0);
    tsch_schedule_add_link(sf_custom, LINK_OPTION_TX, LINK_TYPE_NORMAL, &node_3_address, 4, 0);
  } else if (linkaddr_node_addr.u8[7] == node_2_address.u8[7]) {
    tsch_schedule_add_link(sf_custom, LINK_OPTION_TX, LINK_TYPE_NORMAL, &node_1_address, 1, 0);
    tsch_schedule_add_link(sf_custom, LINK_OPTION_RX, LINK_TYPE_NORMAL, &node_3_address, 3, 0);
    tsch_schedule_add_link(sf_custom, LINK_OPTION_RX, LINK_TYPE_NORMAL, &node_4_address, 5, 0);
    tsch_schedule_add_link(sf_custom, LINK_OPTION_TX, LINK_TYPE_NORMAL, &node_4_address, 6, 0);
  } else if (linkaddr_node_addr.u8[7] == node_3_address.u8[7]) {
    tsch_schedule_add_link(sf_custom, LINK_OPTION_TX, LINK_TYPE_NORMAL, &node_1_address, 2, 0);
    tsch_schedule_add_link(sf_custom, LINK_OPTION_TX, LINK_TYPE_NORMAL, &node_2_address, 3, 0);
    tsch_schedule_add_link(sf_custom, LINK_OPTION_RX, LINK_TYPE_NORMAL, &node_1_address, 4, 0);
  } else if (linkaddr_node_addr.u8[7] == node_4_address.u8[7]) {
    tsch_schedule_add_link(sf_custom, LINK_OPTION_TX, LINK_TYPE_NORMAL, &node_2_address, 5, 0);
    tsch_schedule_add_link(sf_custom, LINK_OPTION_RX, LINK_TYPE_NORMAL, &node_2_address, 6, 0);
  }
}
\end{lstlisting}
